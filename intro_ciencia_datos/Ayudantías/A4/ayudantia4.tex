% Options for packages loaded elsewhere
% Options for packages loaded elsewhere
\PassOptionsToPackage{unicode}{hyperref}
\PassOptionsToPackage{hyphens}{url}
\PassOptionsToPackage{dvipsnames,svgnames,x11names}{xcolor}
%
\documentclass[
  letterpaper,
  DIV=11,
  numbers=noendperiod]{scrartcl}
\usepackage{xcolor}
\usepackage[top=30mm,left=20mm]{geometry}
\usepackage{amsmath,amssymb}
\setcounter{secnumdepth}{5}
\usepackage{iftex}
\ifPDFTeX
  \usepackage[T1]{fontenc}
  \usepackage[utf8]{inputenc}
  \usepackage{textcomp} % provide euro and other symbols
\else % if luatex or xetex
  \usepackage{unicode-math} % this also loads fontspec
  \defaultfontfeatures{Scale=MatchLowercase}
  \defaultfontfeatures[\rmfamily]{Ligatures=TeX,Scale=1}
\fi
\usepackage{lmodern}
\ifPDFTeX\else
  % xetex/luatex font selection
\fi
% Use upquote if available, for straight quotes in verbatim environments
\IfFileExists{upquote.sty}{\usepackage{upquote}}{}
\IfFileExists{microtype.sty}{% use microtype if available
  \usepackage[]{microtype}
  \UseMicrotypeSet[protrusion]{basicmath} % disable protrusion for tt fonts
}{}
\makeatletter
\@ifundefined{KOMAClassName}{% if non-KOMA class
  \IfFileExists{parskip.sty}{%
    \usepackage{parskip}
  }{% else
    \setlength{\parindent}{0pt}
    \setlength{\parskip}{6pt plus 2pt minus 1pt}}
}{% if KOMA class
  \KOMAoptions{parskip=half}}
\makeatother
% Make \paragraph and \subparagraph free-standing
\makeatletter
\ifx\paragraph\undefined\else
  \let\oldparagraph\paragraph
  \renewcommand{\paragraph}{
    \@ifstar
      \xxxParagraphStar
      \xxxParagraphNoStar
  }
  \newcommand{\xxxParagraphStar}[1]{\oldparagraph*{#1}\mbox{}}
  \newcommand{\xxxParagraphNoStar}[1]{\oldparagraph{#1}\mbox{}}
\fi
\ifx\subparagraph\undefined\else
  \let\oldsubparagraph\subparagraph
  \renewcommand{\subparagraph}{
    \@ifstar
      \xxxSubParagraphStar
      \xxxSubParagraphNoStar
  }
  \newcommand{\xxxSubParagraphStar}[1]{\oldsubparagraph*{#1}\mbox{}}
  \newcommand{\xxxSubParagraphNoStar}[1]{\oldsubparagraph{#1}\mbox{}}
\fi
\makeatother

\usepackage{color}
\usepackage{fancyvrb}
\newcommand{\VerbBar}{|}
\newcommand{\VERB}{\Verb[commandchars=\\\{\}]}
\DefineVerbatimEnvironment{Highlighting}{Verbatim}{commandchars=\\\{\}}
% Add ',fontsize=\small' for more characters per line
\usepackage{framed}
\definecolor{shadecolor}{RGB}{241,243,245}
\newenvironment{Shaded}{\begin{snugshade}}{\end{snugshade}}
\newcommand{\AlertTok}[1]{\textcolor[rgb]{0.68,0.00,0.00}{#1}}
\newcommand{\AnnotationTok}[1]{\textcolor[rgb]{0.37,0.37,0.37}{#1}}
\newcommand{\AttributeTok}[1]{\textcolor[rgb]{0.40,0.45,0.13}{#1}}
\newcommand{\BaseNTok}[1]{\textcolor[rgb]{0.68,0.00,0.00}{#1}}
\newcommand{\BuiltInTok}[1]{\textcolor[rgb]{0.00,0.23,0.31}{#1}}
\newcommand{\CharTok}[1]{\textcolor[rgb]{0.13,0.47,0.30}{#1}}
\newcommand{\CommentTok}[1]{\textcolor[rgb]{0.37,0.37,0.37}{#1}}
\newcommand{\CommentVarTok}[1]{\textcolor[rgb]{0.37,0.37,0.37}{\textit{#1}}}
\newcommand{\ConstantTok}[1]{\textcolor[rgb]{0.56,0.35,0.01}{#1}}
\newcommand{\ControlFlowTok}[1]{\textcolor[rgb]{0.00,0.23,0.31}{\textbf{#1}}}
\newcommand{\DataTypeTok}[1]{\textcolor[rgb]{0.68,0.00,0.00}{#1}}
\newcommand{\DecValTok}[1]{\textcolor[rgb]{0.68,0.00,0.00}{#1}}
\newcommand{\DocumentationTok}[1]{\textcolor[rgb]{0.37,0.37,0.37}{\textit{#1}}}
\newcommand{\ErrorTok}[1]{\textcolor[rgb]{0.68,0.00,0.00}{#1}}
\newcommand{\ExtensionTok}[1]{\textcolor[rgb]{0.00,0.23,0.31}{#1}}
\newcommand{\FloatTok}[1]{\textcolor[rgb]{0.68,0.00,0.00}{#1}}
\newcommand{\FunctionTok}[1]{\textcolor[rgb]{0.28,0.35,0.67}{#1}}
\newcommand{\ImportTok}[1]{\textcolor[rgb]{0.00,0.46,0.62}{#1}}
\newcommand{\InformationTok}[1]{\textcolor[rgb]{0.37,0.37,0.37}{#1}}
\newcommand{\KeywordTok}[1]{\textcolor[rgb]{0.00,0.23,0.31}{\textbf{#1}}}
\newcommand{\NormalTok}[1]{\textcolor[rgb]{0.00,0.23,0.31}{#1}}
\newcommand{\OperatorTok}[1]{\textcolor[rgb]{0.37,0.37,0.37}{#1}}
\newcommand{\OtherTok}[1]{\textcolor[rgb]{0.00,0.23,0.31}{#1}}
\newcommand{\PreprocessorTok}[1]{\textcolor[rgb]{0.68,0.00,0.00}{#1}}
\newcommand{\RegionMarkerTok}[1]{\textcolor[rgb]{0.00,0.23,0.31}{#1}}
\newcommand{\SpecialCharTok}[1]{\textcolor[rgb]{0.37,0.37,0.37}{#1}}
\newcommand{\SpecialStringTok}[1]{\textcolor[rgb]{0.13,0.47,0.30}{#1}}
\newcommand{\StringTok}[1]{\textcolor[rgb]{0.13,0.47,0.30}{#1}}
\newcommand{\VariableTok}[1]{\textcolor[rgb]{0.07,0.07,0.07}{#1}}
\newcommand{\VerbatimStringTok}[1]{\textcolor[rgb]{0.13,0.47,0.30}{#1}}
\newcommand{\WarningTok}[1]{\textcolor[rgb]{0.37,0.37,0.37}{\textit{#1}}}

\usepackage{longtable,booktabs,array}
\usepackage{calc} % for calculating minipage widths
% Correct order of tables after \paragraph or \subparagraph
\usepackage{etoolbox}
\makeatletter
\patchcmd\longtable{\par}{\if@noskipsec\mbox{}\fi\par}{}{}
\makeatother
% Allow footnotes in longtable head/foot
\IfFileExists{footnotehyper.sty}{\usepackage{footnotehyper}}{\usepackage{footnote}}
\makesavenoteenv{longtable}
\usepackage{graphicx}
\makeatletter
\newsavebox\pandoc@box
\newcommand*\pandocbounded[1]{% scales image to fit in text height/width
  \sbox\pandoc@box{#1}%
  \Gscale@div\@tempa{\textheight}{\dimexpr\ht\pandoc@box+\dp\pandoc@box\relax}%
  \Gscale@div\@tempb{\linewidth}{\wd\pandoc@box}%
  \ifdim\@tempb\p@<\@tempa\p@\let\@tempa\@tempb\fi% select the smaller of both
  \ifdim\@tempa\p@<\p@\scalebox{\@tempa}{\usebox\pandoc@box}%
  \else\usebox{\pandoc@box}%
  \fi%
}
% Set default figure placement to htbp
\def\fps@figure{htbp}
\makeatother

\ifLuaTeX
  \usepackage{luacolor}
  \usepackage[soul]{lua-ul}
\else
  \usepackage{soul}
\fi




\setlength{\emergencystretch}{3em} % prevent overfull lines

\providecommand{\tightlist}{%
  \setlength{\itemsep}{0pt}\setlength{\parskip}{0pt}}



 


\KOMAoption{captions}{tableheading}
\makeatletter
\@ifpackageloaded{caption}{}{\usepackage{caption}}
\AtBeginDocument{%
\ifdefined\contentsname
  \renewcommand*\contentsname{Table of contents}
\else
  \newcommand\contentsname{Table of contents}
\fi
\ifdefined\listfigurename
  \renewcommand*\listfigurename{List of Figures}
\else
  \newcommand\listfigurename{List of Figures}
\fi
\ifdefined\listtablename
  \renewcommand*\listtablename{List of Tables}
\else
  \newcommand\listtablename{List of Tables}
\fi
\ifdefined\figurename
  \renewcommand*\figurename{Figure}
\else
  \newcommand\figurename{Figure}
\fi
\ifdefined\tablename
  \renewcommand*\tablename{Table}
\else
  \newcommand\tablename{Table}
\fi
}
\@ifpackageloaded{float}{}{\usepackage{float}}
\floatstyle{ruled}
\@ifundefined{c@chapter}{\newfloat{codelisting}{h}{lop}}{\newfloat{codelisting}{h}{lop}[chapter]}
\floatname{codelisting}{Listing}
\newcommand*\listoflistings{\listof{codelisting}{List of Listings}}
\makeatother
\makeatletter
\makeatother
\makeatletter
\@ifpackageloaded{caption}{}{\usepackage{caption}}
\@ifpackageloaded{subcaption}{}{\usepackage{subcaption}}
\makeatother
\usepackage{bookmark}
\IfFileExists{xurl.sty}{\usepackage{xurl}}{} % add URL line breaks if available
\urlstyle{same}
\hypersetup{
  pdftitle={Ayudantía: APIs y Webscraping},
  colorlinks=true,
  linkcolor={blue},
  filecolor={Maroon},
  citecolor={Blue},
  urlcolor={Blue},
  pdfcreator={LaTeX via pandoc}}


\title{Ayudantía: APIs y Webscraping}
\usepackage{etoolbox}
\makeatletter
\providecommand{\subtitle}[1]{% add subtitle to \maketitle
  \apptocmd{\@title}{\par {\large #1 \par}}{}{}
}
\makeatother
\subtitle{IMT2200 2025-2}
\author{}
\date{}
\begin{document}
\maketitle

\renewcommand*\contentsname{Contenido}
{
\hypersetup{linkcolor=}
\setcounter{tocdepth}{3}
\tableofcontents
}

\section{Introducción}\label{introducciuxf3n}

En esta ayudantía veremos de manera introductoria:

\begin{itemize}
\tightlist
\item
  Qué es HTML y porqué nos sirve saber de él.
\item
  Qué son los protocolos y cómo se relacionan con la web.
\item
  Ejemplos prácticos de Webscraping en Python.
\item
  Ejemplo de uso de una API.
\end{itemize}

\begin{center}\rule{0.5\linewidth}{0.5pt}\end{center}

\section{¿Qué es HTML?}\label{quuxe9-es-html}

HTML es el lenguaje de marcado que estructura el contenido de la web, de
ahí vienen los archivos \emph{\texttt{nombre.html}} . Estos archivos son
los que visualizamos a la hora de entrar a cualquier página web, es lo
que los navegadores procesan y muestran en nuestras pantallas.

Este lenguaje tiene una estructura y una lógica. No es necesario que
conozcamos todas sus características, pero sí entendamos cómo se
comporta.

Los bloques que construyen un \emph{\texttt{archivo.html}} son las
\textbf{etiquetas/tags} (eg:
\texttt{\textless{}h1\textgreater{}},\texttt{\textless{}p\textgreater{}},\texttt{\textless{}div\textgreater{}},
etc), cada una de estas es interpretada como un distinto tipo de
información o bloque y contienen características intrínsecas (que pueden
ser modificadas).

Por ejemplo :

\begin{Shaded}
\begin{Highlighting}[]
\NormalTok{\textless{}h1\textgreater{}Hola mundo\textless{}/h1\textgreater{}}
\NormalTok{\textless{}p\textgreater{}Este es un párrafo con un \textless{}a href="https://www.wikipedia.org"\textgreater{}link\textless{}/a\textgreater{}.\textless{}/p\textgreater{}}
\end{Highlighting}
\end{Shaded}

Se renderizaría en una página web así :

Hola mundo

Este es un párrafo con un link.

\begin{center}\rule{0.5\linewidth}{0.5pt}\end{center}

Podemos hacernos una idea visual del comportamiento de los elementos
como sigue:

\begin{figure}[H]

{\centering \includegraphics[width=0.5\linewidth,height=0.5\textheight]{imgs/html.png}

}

\caption{Esquema HTML}

\end{figure}%

Es bueno pensar que cada etiqueta \textbf{html} es como un objeto de
\textbf{python}, y por lo tanto tiene atributos (que son las cosas que
podemos editar), los que nos sirve saber para este curso :

\begin{itemize}
\item
  \texttt{id}: Identificador único de algún elemento.
\item
  \texttt{class}: Clase o categoría a la que pertenece el elemento, por
  ejemplo: \texttt{class="img-pequenas"}.
\item
  \texttt{style}: Estilos CSS aplicados al elemento. Aquí podemos
  modificar márgenes, anchos, alturas, relleno, colores, etc. Es todo
  otro lenguaje, que se escapa del alcance del curso.
\end{itemize}

Ahora, que conocemos el material con el que se trabaja en la web, veamos
cómo se comunican los diferentes agentes de este proceso, y que hacen
posible la existencia del internet.

\section{Protocolos: HTTP y HTTPS}\label{protocolos-http-y-https}

Para que sea posible acceder a una página web, dependiendo de su
contenido, pasan una serie de procesos y protocolos. Tenemos
\textbf{HTTP} (HyperText Transfer Protocol) y su versión segura
\textbf{HTTPS} (HTTP Secure).

Podemos decir que este protocolo se compone de los siguientes pasos:

\begin{itemize}
\tightlist
\item
  Tu navegador (cliente) envía una solicitud (request).(¿A quién? 🧐)
\item
  Digamos que a otro computador, el servidor le llamaremos, quien recibe
  esa señal, dependiendo del mensaje, da una respuesta.
\item
  la respuesta llega, y el navegador la procesa para mostrarla al
  usuario.
\end{itemize}

(esta ejemplificación está muy simplificada, pero es la idea general)

\begin{figure}[H]

{\centering \includegraphics[width=0.5\linewidth,height=0.5\textheight]{imgs/cliente.png}

}

\caption{Arquitectura Cliente-Servidor}

\end{figure}%

\subsubsection{Las solicitudes (HTTP
Requests)}\label{las-solicitudes-http-requests}

Las solicitudes HTTP son la forma en que los navegadores y otros
clientes se comunican con los servidores. Estas anticipan la naturaleza
de la interacción que se desea tener con el servidor. Las más relevantes
son:

\begin{itemize}
\tightlist
\item
  \textbf{GET}: Solicita datos del servidor. Es la más común y se
  utiliza para obtener información.
\item
  \textbf{POST}: Envía datos al servidor. Se utiliza para enviar
  información, como formularios.
\item
  \textbf{PUT}: Actualiza datos en el servidor.
\item
  \textbf{DELETE}: Elimina datos del servidor.
\end{itemize}

\ul{👉 \textbf{Ejemplo:}}

Cuando escribes \texttt{https://www.google.com}, tu navegador hace un
GET al servidor de Google. Google recibe la solicitud, la revisa y
responde con su página de inicio. Esto finalmente se muestra en tu
navegador.

\begin{center}\rule{0.5\linewidth}{0.5pt}\end{center}

\section{Webscraping}\label{webscraping}

¿De que nos sirve saber todo esto? En ciencia de datos, muchas veces
queremos extraer información de páginas web. Entonces, ahora que ya
conocemos, a grandes rasgos, el comportamiento y estructura de la web,
podemos entender mejor los procesos que son necesarios para el
Webscraping.

Pero ¿qué es el Webscraping? Es la técnica utilizada para extraer
información de sitios web de manera automatizada (cuando no podemos
obtener la información de otra forma). En esta ocasión ocuparemos
algunas librerias de python para hacer el scrapping HTML de las páginas
web.

Veamos un pequeño ejemplo :

\begin{Shaded}
\begin{Highlighting}[]
\ImportTok{import}\NormalTok{ requests}
\ImportTok{from}\NormalTok{ bs4 }\ImportTok{import}\NormalTok{ BeautifulSoup}

\NormalTok{url }\OperatorTok{=} \StringTok{"https://quotes.toscrape.com/"}
\NormalTok{response }\OperatorTok{=}\NormalTok{ requests.get(url) }\CommentTok{\# Hacemos la solicitud GET}
\BuiltInTok{print}\NormalTok{(response.status\_code) }\CommentTok{\# 200 es que todo salió bien}
\BuiltInTok{print}\NormalTok{(response)}
\end{Highlighting}
\end{Shaded}

\begin{verbatim}
200
<Response [200]>
\end{verbatim}

\textbf{¿Qué esta pasando?}

\texttt{response.text} nos da el HTML (de la página que solicitamos)
como un string. Luego usaremos \emph{BeautifulSoup} para transformar ese
string de HTML en una estructura más manejable.

\begin{Shaded}
\begin{Highlighting}[]
\NormalTok{bs }\OperatorTok{=}\NormalTok{ BeautifulSoup(response.text, }\StringTok{\textquotesingle{}html.parser\textquotesingle{}}\NormalTok{) }\CommentTok{\# Parseamos el HTML}
\NormalTok{quotes }\OperatorTok{=}\NormalTok{ [q.get\_text() }\ControlFlowTok{for}\NormalTok{ q }\KeywordTok{in}\NormalTok{ bs.select(}\StringTok{".quote span.text"}\NormalTok{)]}
\NormalTok{quotes[:}\DecValTok{5}\NormalTok{]}
\end{Highlighting}
\end{Shaded}

\begin{verbatim}
['“The world as we have created it is a process of our thinking. It cannot be changed without changing our thinking.”',
 '“It is our choices, Harry, that show what we truly are, far more than our abilities.”',
 '“There are only two ways to live your life. One is as though nothing is a miracle. The other is as though everything is a miracle.”',
 '“The person, be it gentleman or lady, who has not pleasure in a good novel, must be intolerably stupid.”',
 "“Imperfection is beauty, madness is genius and it's better to be absolutely ridiculous than absolutely boring.”"]
\end{verbatim}

\begin{center}\rule{0.5\linewidth}{0.5pt}\end{center}

Pero lo que buscamos es información específica.Con \texttt{.select()}
buscamos dentro del HTML todos los elementos que cumplan con lo
especificado en los argumentos.

En este caso:

\texttt{.quote} significa ``cualquier elemento con la clase quote'' (las
clases se designan con un ``\textbf{.}'').

\texttt{span.text} significa ``dentro de eso, los
\texttt{\textless{}span\textgreater{}} que tengan la clase text''.

Esto nos devuelve una lista de esos elementos solicitados. Luego
simplemente imprimimos los primeros 5 para verificar que se ejecutó
correctamente nuestro código.

\section{APIs}\label{apis}

Como ya vieron en clases, también existen las APIs (Application
Programming Interfaces) que nos permiten obtener datos de forma
estructurada, sin necesidad de webscraping. Las APIs también requieren
cierto protocolo y estructura, dependen sobre todo del proveedor.
Algunas son de libre acceso, otras pagadas, otras requieren
autenticación, etc.

Ejemplo manejo de una API pública (Pokémon):

\begin{Shaded}
\begin{Highlighting}[]
\ImportTok{import}\NormalTok{ requests}

\NormalTok{url }\OperatorTok{=} \StringTok{"https://pokeapi.co/api/v2/pokemon/pikachu"}
\NormalTok{data }\OperatorTok{=}\NormalTok{ requests.get(url).json() }\CommentTok{\# pasamos la respuesta a JSON}
\NormalTok{data[}\StringTok{"abilities"}\NormalTok{] }\CommentTok{\# JSON podemos acceder a los atributos como si fuera un diccionario (tiene estructura clave:valor)}
\end{Highlighting}
\end{Shaded}

\begin{verbatim}
[{'ability': {'name': 'static', 'url': 'https://pokeapi.co/api/v2/ability/9/'},
  'is_hidden': False,
  'slot': 1},
 {'ability': {'name': 'lightning-rod',
   'url': 'https://pokeapi.co/api/v2/ability/31/'},
  'is_hidden': True,
  'slot': 3}]
\end{verbatim}

Para una API que requiere API Key, podemos hacer lo sgte :

\begin{Shaded}
\begin{Highlighting}[]
\ImportTok{import}\NormalTok{ requests}
\ImportTok{import}\NormalTok{ datetime}

\NormalTok{API\_KEY }\OperatorTok{=} \StringTok{"DEMO\_KEY"}
\NormalTok{date }\OperatorTok{=}\NormalTok{ datetime.date.today() }\OperatorTok{{-}}\NormalTok{ datetime.timedelta(days}\OperatorTok{=}\DecValTok{1}\NormalTok{)}

\NormalTok{url }\OperatorTok{=} \SpecialStringTok{F"https://api.nasa.gov/planetary/apod?api\_key=}\SpecialCharTok{\{}\NormalTok{API\_KEY}\SpecialCharTok{\}}\SpecialStringTok{\&date=}\SpecialCharTok{\{}\NormalTok{date}\SpecialCharTok{\}}\SpecialStringTok{"}
\NormalTok{headers }\OperatorTok{=}\NormalTok{ \{}
    \StringTok{"Authorization"}\NormalTok{: }\SpecialStringTok{f"Bearer }\SpecialCharTok{\{}\NormalTok{API\_KEY}\SpecialCharTok{\}}\SpecialStringTok{"}
\NormalTok{\}}
\NormalTok{data }\OperatorTok{=}\NormalTok{ requests.get(url, headers}\OperatorTok{=}\NormalTok{headers).json()}
\NormalTok{data}
\end{Highlighting}
\end{Shaded}

\begin{center}\rule{0.5\linewidth}{0.5pt}\end{center}

Es muy similar a la anterior, pero ahora el proveedor nos esta pidiendo
una autenticación antes de darnos acceso a los datos. Sin esa
autenticación no podemos hacer nada. Esta es una forma común de proteger
las APIs y asegurar que solo los usuarios autorizados puedan acceder a
los datos ( y también modificarlos, según su nivel de acceso).

\begin{figure}[H]

{\centering \includegraphics[width=0.5\linewidth,height=0.5\textheight]{imgs/api.avif}

}

\caption{Arquitectura API}

\end{figure}%




\end{document}
